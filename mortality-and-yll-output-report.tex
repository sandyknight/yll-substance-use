% Options for packages loaded elsewhere
\PassOptionsToPackage{unicode}{hyperref}
\PassOptionsToPackage{hyphens}{url}
%
\documentclass[
]{article}
\usepackage{amsmath,amssymb}
\usepackage{lmodern}
\usepackage{iftex}
\ifPDFTeX
  \usepackage[T1]{fontenc}
  \usepackage[utf8]{inputenc}
  \usepackage{textcomp} % provide euro and other symbols
\else % if luatex or xetex
  \usepackage{unicode-math}
  \defaultfontfeatures{Scale=MatchLowercase}
  \defaultfontfeatures[\rmfamily]{Ligatures=TeX,Scale=1}
\fi
% Use upquote if available, for straight quotes in verbatim environments
\IfFileExists{upquote.sty}{\usepackage{upquote}}{}
\IfFileExists{microtype.sty}{% use microtype if available
  \usepackage[]{microtype}
  \UseMicrotypeSet[protrusion]{basicmath} % disable protrusion for tt fonts
}{}
\makeatletter
\@ifundefined{KOMAClassName}{% if non-KOMA class
  \IfFileExists{parskip.sty}{%
    \usepackage{parskip}
  }{% else
    \setlength{\parindent}{0pt}
    \setlength{\parskip}{6pt plus 2pt minus 1pt}}
}{% if KOMA class
  \KOMAoptions{parskip=half}}
\makeatother
\usepackage{xcolor}
\usepackage[margin=1in]{geometry}
\usepackage{graphicx}
\makeatletter
\def\maxwidth{\ifdim\Gin@nat@width>\linewidth\linewidth\else\Gin@nat@width\fi}
\def\maxheight{\ifdim\Gin@nat@height>\textheight\textheight\else\Gin@nat@height\fi}
\makeatother
% Scale images if necessary, so that they will not overflow the page
% margins by default, and it is still possible to overwrite the defaults
% using explicit options in \includegraphics[width, height, ...]{}
\setkeys{Gin}{width=\maxwidth,height=\maxheight,keepaspectratio}
% Set default figure placement to htbp
\makeatletter
\def\fps@figure{htbp}
\makeatother
\setlength{\emergencystretch}{3em} % prevent overfull lines
\providecommand{\tightlist}{%
  \setlength{\itemsep}{0pt}\setlength{\parskip}{0pt}}
\setcounter{secnumdepth}{5}
\ifLuaTeX
  \usepackage{selnolig}  % disable illegal ligatures
\fi
\IfFileExists{bookmark.sty}{\usepackage{bookmark}}{\usepackage{hyperref}}
\IfFileExists{xurl.sty}{\usepackage{xurl}}{} % add URL line breaks if available
\urlstyle{same} % disable monospaced font for URLs
\hypersetup{
  pdftitle={Years of life lost due to substance misuse},
  pdfauthor={Sandy Knight},
  hidelinks,
  pdfcreator={LaTeX via pandoc}}

\title{Years of life lost due to substance misuse}
\author{Sandy Knight}
\date{2024-12-07}

\begin{document}
\maketitle

{
\setcounter{tocdepth}{2}
\tableofcontents
}
\hypertarget{deaths-associated-with-drug-use}{%
\subsubsection{Deaths associated with drug
use}\label{deaths-associated-with-drug-use}}

The ONS publishes drug poisonings and classifies them as ``related to
drug misuse'' given certain criteria i.e.~specific ICD-10 codes on the
death record.

The linkage of NDTMS and ONS data reveals that some deaths that had
insufficient information for the ONS to apply this classification are
probably related to drug misuse since they occured either in treatment
or within a year of discharge. This analysis uses data from both sources
to adjust the total number of deaths to include these additional deaths.

The NDTMS-ONS data linkage also has records for people who died in
treatment or after treatment from causes other than drug poisoning. This
analysis adds those deaths to either drug- or alcohol-related deaths
depending on the cause and, in the case of those people in treatment for
drug use, time since discharge.

This analysis aggregates and plots the total count of deaths associated
with drug use. The only categories are the substance, poisoning or
non-poisoning, and whether they were counted in the initial ONS data or
added from analysis of the linked dataset:

\begin{itemize}
\tightlist
\item
  Initial poisoning deaths
\item
  Additional poisoning deaths
\item
  Non-poisoning deaths: Died in treatment
\item
  Non-poisoning deaths: Died within a year of discharge
\item
  Non-poisoning deaths: Died one or more years following discharge
  \emph{(not counted in the total)}
\end{itemize}

\includegraphics{mortality-and-yll-output-report_files/figure-latex/unnamed-chunk-31-1.pdf}

\hypertarget{deaths-associated-with-alcohol}{%
\subsubsection{Deaths associated with
alcohol}\label{deaths-associated-with-alcohol}}

Using the same method with the ONS alcohol-specific deaths data and the
linked dataset we can identify additional deaths related to alcohol
deaths.

\includegraphics{mortality-and-yll-output-report_files/figure-latex/unnamed-chunk-32-1.pdf}

\hypertarget{deaths-associated-with-substance-use}{%
\subsubsection{Deaths associated with substance
use}\label{deaths-associated-with-substance-use}}

We can then combine the counts of additional deaths associated with
substance use and see the estimated total impact of substance use on
mortality (not including ``alcohol-related deaths'' as defined by the
ONS, as they are mostly not the population in treatment or with a
treatment need)

\includegraphics{mortality-and-yll-output-report_files/figure-latex/unnamed-chunk-33-1.pdf}

\hypertarget{years-of-life-lost}{%
\subsubsection{Years of life lost}\label{years-of-life-lost}}

Using the counts of deaths associated with drug misuse stratified by age
and the
\href{https://www.ons.gov.uk/peoplepopulationandcommunity/birthsdeathsandmarriages/lifeexpectancies/datasets/nationallifetablesunitedkingdomreferencetables}{life
expectancy tables from the ONS} we can estimate the years of life lost
(YLL) due to substance use.

Just using the age (or age group) of death and the mean life expectancy
for that age we can produce a YLL estimate for drugs and alcohol.

Expected years of life lost with this method:

\[
YLL = (D_{x})(e_{x}^s)
\]

Where \(D_x\) is the number of deaths in the age group and \(e_{x}^s\)
is the mean standard age of death from the external life expectancy (the
ONS life tables) for the age.

\includegraphics{mortality-and-yll-output-report_files/figure-latex/unnamed-chunk-34-1.pdf}

However when YLL was incorporated into the Global Burden of Disease
study (GBD) \footnote{Murray, C.J., Lopez, A.D. and Jamison, D.T., 1994.
  The global burden of disease in 1990: summary results, sensitivity
  analysis and future directions. \emph{Bulletin of the world health
  organization}, 72(3), p.495.} and the authors determined that
discounting and age-weighting should be applied to better measure
premature mortality. We use the GBD formula and parameters and apply
them using open-source code published by Aragón, T.J. et al. \footnote{Aragón,
  T.J., Lichtensztajn, D.Y., Katcher, B.S., Reiter, R. and Katz, M.H.,
  2007. Calculating Expected Years of Life Lost to Rank the Leading
  Causes of Premature Death in San Francisco. \emph{San Francisco
  Department of Public Health}.}

The formula is:

\[Y_x = d_x \left[ \frac{KCe^{r(n^a_x)}}{(r+\beta)^2} \left( e^{z[-(r+\beta)(e^s_x + a_x) - 1]} - e^{-(r+\beta)a_x[-(r+\beta)a_x - 1]} \right) + \frac{1-K}{r} (1 - e^{r(e^s_x)}) \right]\]

~\\
where:\\
𝑎 = age of death (years)\\
𝑟 = discount rate (usually 3\%)\\
𝛽 = age weighting constant (usually 𝛽=0.04)\\
𝐾 = age-weighting modulation constant (usually 𝐾=1)\\
𝐶 = adjustment constant for age-weights (usually 𝐶=0.1658)\\
𝑒 = standard life expectancy at age of death (years)\\

The default values for these parameters were chosen and calibrated in
the original Global Burden of Disease (GBD) study \footnote{Murray,
  C.J., Lopez, A.D. and Jamison, D.T., 1994. The global burden of
  disease in 1990: summary results, sensitivity analysis and future
  directions. \emph{Bulletin of the world health organization}, 72(3),
  p.495.}

\includegraphics{mortality-and-yll-output-report_files/figure-latex/unnamed-chunk-35-1.pdf}

\hypertarget{comparison-with-leading-causes-of-mortality}{%
\subsubsection{Comparison with leading causes of
mortality}\label{comparison-with-leading-causes-of-mortality}}

We also compared the augmented count of deaths associated with substance
use with the leading causes of mortality
\href{https://www.ons.gov.uk/peoplepopulationandcommunity/healthandsocialcare/causesofdeath/datasets/avoidablemortalitybylocalauthorityinenglandandwales}{data}
from the ONS.

When stratified by age group and the inclusion of drugs, alcohol, both,
or the sum of both, this produces sixteen plots comparing substance use
deaths with the leading causes of mortality:

\begin{verbatim}
## [[1]]
\end{verbatim}

\includegraphics{mortality-and-yll-output-report_files/figure-latex/unnamed-chunk-36-1.pdf}

\begin{verbatim}
## 
## [[2]]
\end{verbatim}

\includegraphics{mortality-and-yll-output-report_files/figure-latex/unnamed-chunk-36-2.pdf}

\begin{verbatim}
## 
## [[3]]
\end{verbatim}

\includegraphics{mortality-and-yll-output-report_files/figure-latex/unnamed-chunk-36-3.pdf}

\begin{verbatim}
## 
## [[4]]
\end{verbatim}

\includegraphics{mortality-and-yll-output-report_files/figure-latex/unnamed-chunk-36-4.pdf}

\begin{verbatim}
## 
## [[5]]
\end{verbatim}

\includegraphics{mortality-and-yll-output-report_files/figure-latex/unnamed-chunk-36-5.pdf}

\begin{verbatim}
## 
## [[6]]
\end{verbatim}

\includegraphics{mortality-and-yll-output-report_files/figure-latex/unnamed-chunk-36-6.pdf}

\begin{verbatim}
## 
## [[7]]
\end{verbatim}

\includegraphics{mortality-and-yll-output-report_files/figure-latex/unnamed-chunk-36-7.pdf}

\begin{verbatim}
## 
## [[8]]
\end{verbatim}

\includegraphics{mortality-and-yll-output-report_files/figure-latex/unnamed-chunk-36-8.pdf}

\begin{verbatim}
## 
## [[9]]
\end{verbatim}

\includegraphics{mortality-and-yll-output-report_files/figure-latex/unnamed-chunk-36-9.pdf}

\begin{verbatim}
## 
## [[10]]
\end{verbatim}

\includegraphics{mortality-and-yll-output-report_files/figure-latex/unnamed-chunk-36-10.pdf}

\begin{verbatim}
## 
## [[11]]
\end{verbatim}

\includegraphics{mortality-and-yll-output-report_files/figure-latex/unnamed-chunk-36-11.pdf}

\begin{verbatim}
## 
## [[12]]
\end{verbatim}

\includegraphics{mortality-and-yll-output-report_files/figure-latex/unnamed-chunk-36-12.pdf}

\begin{verbatim}
## 
## [[13]]
\end{verbatim}

\includegraphics{mortality-and-yll-output-report_files/figure-latex/unnamed-chunk-36-13.pdf}

\begin{verbatim}
## 
## [[14]]
\end{verbatim}

\includegraphics{mortality-and-yll-output-report_files/figure-latex/unnamed-chunk-36-14.pdf}

\begin{verbatim}
## 
## [[15]]
\end{verbatim}

\includegraphics{mortality-and-yll-output-report_files/figure-latex/unnamed-chunk-36-15.pdf}

\begin{verbatim}
## 
## [[16]]
\end{verbatim}

\includegraphics{mortality-and-yll-output-report_files/figure-latex/unnamed-chunk-36-16.pdf}

\end{document}
